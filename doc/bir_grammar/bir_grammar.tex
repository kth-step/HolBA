\documentclass{article}
\usepackage{syntax}

\begin{document}
\section{BIR syntax}

\begin{itemize}
\item \syntax{<literal>} is a numeric literal, either in decimal notation (e.g. 123)
or in hexadecimal (e.g. 0xabc).

\item \syntax{<var>} is a variable name.

\item Variables that begin with the prefix ``MEM'' will be parsed as memory
  variables of default memory type; these are the only allowed memory variables,
  and they cannot be treated as registers. Any keywords that appear in the
  grammar below are reserved and cannot occur as variables. The string ``__ite''
  is also reserved and used internally by the parser.
\end{itemize}

\setlength{\grammarparsep}{4pt plus 1pt minus 1pt} % increase separation between rules
\setlength{\grammarindent}{12em} % increase separation between LHS/RHS 

\begin{grammar}

  <bit-type> ::= Bit8 | Bit16 | Bit32 | Bit64

  <cast-op> ::= ucast | scast | hcast | lcast

  <unary-op> ::= chsign | cls | clz

  <infix-op> ::= + | - | * | / | \% | $\ll$ | $\gg$ | == | ...

  <prefix-op> ::= sdiv | smod | memeq | srsh | slt | sle

  <expr> ::= <var> \alt <literal>
  \alt \textasciitilde <expr>
  \alt <unary-op> <expr>
  \alt <expr> <infix-op> <expr>
  \alt <prefix-op> <expr> <expr>
  \alt ld <expr> <expr>
  \alt ld\{8,16,32,64\} <expr> <expr>
  \alt st <expr> <expr> <expr>
  \alt st\{8,16,32,64\} <expr> <expr> <expr>
  \alt <cast-op> <expr> <bit-type>
  \alt if <expr> then <expr> else <expr>
  \alt <expr>:<bit-type>

  <statement> ::= assert <expr>
  \alt assume <expr>
  \alt observe <literal> <expr> <expr>
  \alt <var> = <expr>

  <end-statement> ::= halt <expr>
  \alt jmp <expr>
  \alt cjmp <expr> <expr> <expr>

  <stat-list> ::= <statement> `\\n' <stat-list> | <statement> 

  <label> ::= <string> | <literal>

  <block> ::= <label>: `\\n' <stat-list> `\\n' <end-statement>

  <program> ::= <block> `\\n' <program> | <block>
\end{grammar}

\section{Usage notes}

The quotation library can be found in \texttt{bir_quotationLib} in the HolBA
shared heap.

\subsection{Parser}

\paragraph{Invoking the quotation parser} Two parsers are provided:
\begin{enumerate}
\item \texttt{BExp}, to parse a single BIR expression, e.g. \texttt{BExp`X1+4`}; and
\item \texttt{BIR}, to parse an entire BIR program.
\end{enumerate}

Here is an example of the latter:

\begin{verbatim}
val prog =
BIR‘
a:
assert (X1 > 0)
X1 = if X2 == 3 then 2 else 3
MEM = st MEM 0x4 (sdiv 100 8)
jmp b
b:
halt X1
’;
\end{verbatim}

Single-line comments are supported in the \texttt{BIR} parser only and begin
with \texttt{';'}. Any characters that follow will be ignored up to the next
newline. Example:

\begin{verbatim}
val prog_with_comments =
BIR‘
a:
assert (X1 > 0)
X1 = X2 + X3 ; this is a comment
jmp b ;another comment
b:
; can also be stand-alone
halt X1
’;
\end{verbatim}


\paragraph{Type annotations} The parser assumes all variables and literals to be
of 64-bit word type (\syntax{Bit64}). To explicitly ascribe a different type, we
can use operator \texttt{:} as follows:

\begin{verbatim}
> BExp`5 : Bit32`;
val it = ``BExp_Const (Imm32 5w)``: term
\end{verbatim}

Without the explicit annotation we get

\begin{verbatim}
> BExp`5`;
val it = ``BExp_Const (Imm64 5w)``: term
\end{verbatim}

Type annotations can be applied to an entire expression, causing all variables
and literals to be interpreted at the explicitly-provided type. For instance

\begin{verbatim}
> BExp`(5+5):Bit32`;
val it = ``BExp_BinExp BIExp_Plus
   (BExp_Const (Imm32 5w)) (BExp_Const (Imm32 5w))``: term
\end{verbatim}


Note that this operator is not a cast, just a hint to the parser.

\paragraph{Memory operation defaults} All memory operations assume that the
memory maps 64-bit addresses to 8-bit words, and they are also assumed to be
little endian. The default memory sizes can be controlled with global flags
\texttt{quotation_default_size} and
\texttt{quotation_default_size_byte}.

\paragraph{Load size} Loads default to 64-bit, i.e. loading 8 consecutive bytes
from the specified address. This behaviour can be overridden by using the
explicit type annotation operator \texttt{:}, or by using one of the forms with
a specific size. For example, \texttt{ld16} will load a 16-bit value.

\begin{verbatim}
> BExp`ld16 MEM 0x4000`
  ``BExp_Load (BExp_Den (BVar "MEM" (BType_Mem Bit64 Bit8)))
      (BExp_Const (Imm64 16384w)) BEnd_LittleEndian Bit16``: term
\end{verbatim}

\paragraph{Store size}
Stores automatically use the size of the value being stored. If needed, it's
possible to override the default size for parsing the value by using the special
\texttt{st} forms with a specific size. For example, \texttt{st8 MEM 0x4000 42}
will store a single byte, and it is equivalent to writing \texttt{st MEM 0x4000
  (42:Bit8)}.

\paragraph{Observations} Observations are assumed to be of \texttt{bir_val_t}
type, and the observation function is fixed to \texttt{`HD`}.

\paragraph{Whitespace} Extra whitespace between tokens generally doesn't matter.
For example \texttt{BExp`1+1`} and \texttt{BExp`1 + 1`} are equivalent.

\paragraph{Statement separators} Statements in a block are separated by
newlines. The label of a block should also be on a separate line.

\paragraph{Application syntax} All multi-operand forms use curried application
syntax, e.g. \texttt{BExp`ld MEM (X2+0x10)`}, \texttt{BExp`ucast X1 Bit32`}.

\newpage

\subsection{Pretty-printer}

\paragraph{Invoking the pretty printer} Two pretty printers are provided:
\begin{enumerate}
\item \texttt{expr_to_string : term -> string}, to pretty-print a single BIR expression, e.g.
  \texttt{expr_to_string (BExp`X1+4`) = "X1 + 4"}; and
\item \texttt{prog_to_string : term -> string}, to pretty-print an entire BIR program.
\end{enumerate}

These functions should be ``inverses'' of \texttt{BExp} and \texttt{BIR} modulo
whitespace, comments, unnecessary brackets, associativity of infix operators,
canonical forms of arithmetic comparisons, and the base of numeric literals,
which are always printed in decimal notation. This property has not been
verified or tested thoroughly.

Some examples where the inverse doesn't hold in the strict sense:

\begin{verbatim}
> expr_to_string (BExp `X1 > 0x1000`);
val it = "4096 < X1": string
> expr_to_string (BExp `(X1 + X2) + X3`);
val it = "X1 + X2 + X3": string
\end{verbatim}

\end{document}